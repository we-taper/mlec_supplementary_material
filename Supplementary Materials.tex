\documentclass{article}
\usepackage[utf8]{inputenc}

\title{Supplementary Materials}
\usepackage{graphicx}
\usepackage{fancyvrb}
\usepackage{hyperref}
\hypersetup{ % makes hyperref's links friendly to read
    colorlinks=true,
    linkcolor=blue,
    filecolor=blue,   
    urlcolor=blue,
}

\usepackage[numbers,sort&compress]{natbib}
\renewcommand{\citet}[1]{Ref.~\cite{#1}}


\begin{document}

\maketitle

In this supplementary material, we provide the details necessary to reproduce experimental results from codes $[[5,1,2]]$, $[[6,3,2]]$, $[[8,3,2]]$, and $[[10,1,2]]$. For each code these details comprise:
\begin{itemize}
    \item The parity check matrix. We follows the standard convention of parity check matrix (see e.g.~\citet{gottesman1997stabilizer}) where the X checks come first and the Z checks come second.
    \item The encoding circuit $E$ 
    \item The stabilizer measurement circuits
    \item The correction circuit which projects quantum states back to the code space
    \item The circuit implementations of logical gates
\end{itemize}

For convenience, a gate $g$ acting on $a,b,\cdots$ qubits is denoted as $g[a,b,\cdots]$. Also, each of the logical gates we provide may contain extra Pauli gates acting on the logical qubits. However, one can easily remove these spurious gates by conjugating with the relevant logical Pauli gate. Finally, all circuits are provided in the OpenQASM format~\cite{1707.03429}.

\tableofcontents

\newcommand{\includefilenamed}[2]{
% \VerbatimInput[frame=single,label=#1,labelposition=topline,framesep=2mm]{#2}
\subsection{#1}
\VerbatimInput{#2}
}

\section{Code [[5,1,2]]}
For the code [[5,1,2]], we provide the implementations of $S^{\dagger}$ and $ZS^{\dagger}H$, from which the reader can generate the implementations of $S$ and $H$. (Note, the logical Z gate is $Z[1,2,5]$).

\includefilenamed{The parity check matrix}{512/5_1_2_parity_check_matrix.txt}
\includefilenamed{The encoding circuit $E$}{512/5_1_2_encoding.qasm}
\includefilenamed{The correction circuit $C$}{512/5_1_2_corrector.qasm}
\includefilenamed{The check circuit}{512/5_1_2_check.qasm}
\includefilenamed{The circuit for logical $S^\dagger$}{512/5_1_2_Sdagger_phyOp.qasm}
\includefilenamed{The circuit for logical $Z S^\dagger H$}{512/5_1_2_ZSdaggerH_phyOp.qasm}

\section{Code [[6,3,2]]}
For the code [[6,3,2]], we provide the implementation of $H[1,2]CZ[1,2]H[1,2]$.

\includefilenamed{The parity check matrix}{632/6_3_2_parity_check_matrix.txt}
\includefilenamed{The encoding circuit $E$}{632/6_3_2_encoding.qasm}
\includefilenamed{The correction circuit $C$}{632/6_3_2_corrector.qasm}
\includefilenamed{The check circuit}{632/6_3_2_check.qasm}
\includefilenamed{The circuit for logical $H[1,2]CZ[1,2]H[1,2]$}{632/6_3_2_H[1,2]CZ[1,2]H[1,2]_phyOp.qasm}

\section{Code [[8,3,2]]}
For the code [[8,3,2]], we provide the implementations of $CZ[1,2]$, $CZ[2,3]$, and $CZ[1,3]$, from which the reader can generate the implementations of $S$ and $H$.


\includefilenamed{The parity check matrix}{832/8_3_2_parity_check_matrix.txt}
\includefilenamed{The encoding circuit $E$}{832/8_3_2_encoding.qasm}
\includefilenamed{The correction circuit $C$}{832/8_3_2_corrector.qasm}
\includefilenamed{The check circuit}{832/8_3_2_check.qasm}
\includefilenamed{The circuit for logical $CZ[1,2]$}{832/8_3_2_H[1,2]CZ[1,2]H[1,2]_phyOp.qasm}  % T=1566497665
\includefilenamed{The circuit for logical $CZ[2,3]$}{832/8_3_2_H[2,3]CZ[2,3]H[2,3]_phyOp.qasm}  % T=1566838029
\includefilenamed{The circuit for logical $CZ[1,3]$}{832/8_3_2_H[1,2,3]CZ[1,3]H[1,2,3]_phyOp.qasm}  % T=1566827337

\section{Code [[10, 1, 2]]}

\includefilenamed{The parity check matrix}{1012/10_1_2_parity_check_matrix.txt}
\includefilenamed{The encoding circuit $E$}{1012/10_1_2_encoding.qasm}
\includefilenamed{The correction circuit $C$}{1012/10_1_2_corrector.qasm}
\includefilenamed{The check circuit}{1012/10_1_2_check.qasm}
\includefilenamed{The circuit for logical $-S^\dagger$}{1012/1576012450.3874469_phyOp.qasm}  % T=1576012450.3874469
\includefilenamed{The circuit for logical $-T^\dagger$}{1012/1575986852.1311133_phyOp.qasm}  % T=1575986852.1311133

\bibliographystyle{utphys.mod}
% \bibliographystyle{plain}
\bibliography{refs}

\end{document}


% useful script: s.replace(',', '{,}').replace('_', '\_').replace(']','{]}')